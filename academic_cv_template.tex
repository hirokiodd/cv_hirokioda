
\documentclass[11pt, a4paper]{article}
\usepackage[utf8]{inputenc}
\usepackage{geometry}
\usepackage{enumitem}

% Set page margins
\geometry{left=2.5cm, top=2.5cm, right=2.5cm, bottom=2.5cm}

\begin{document}

% Header
\begin{center}
{\Large \textbf{Hiroki Oda}} \\
\textit{Department of Methodology, London School of Economics and Political Science} \\
\texttt{h.oda@lse.ac.uk | www.yourwebsite.com} \\
\end{center}

% Education
\section*{Education}
2020 \textbf{The University of Tokyo, Department of Sociology} BA Sociology \\
GPA: 3.64 \\
\textit{Modules} \\
\begin{itemize}[leftmargin=*]
    \item Principle of Sociology - Introductory lectures on sociology \\
    \item Research Methods in Sociology - Focused on quantitative methods \\
    \item Seminar in Sociology - Quantitative analysis of Japanese social stratification \\
    \item History of Sociological Theories - Introduction of sociological theories \\
    \item Topics in Sociology - Sociology of science technology and research in action \\
    \item Dissertation - Educational Assortative Mating in Japan (Described in Research Experience) \\
\end{itemize}

The degree in Sociology has given me skills of quantitative research using SPSS and R, critical thinking towards existing theory and common sense, logical thinking and discussion in abstract and concrete ideas. A dissertation for BA is awarded as one of the best dissertations of the year.

\textbf{Program outside of the department}, Global Education for Innovation and Leadership (GEfIL) Program \\
\textit{Modules} \\
\begin{itemize}[leftmargin=*]
    \item Independent Research Project – Research on Rohingya refugees in Japan \\
    \item Global Leader Lecture Series – Lectures on problem-solving strategies by globally engaged speakers \\
\end{itemize}

The honours program at The University of Tokyo has given me the skill of doing acdemic research in English, the ability in doing projects with people from different backgrounds and transdisciplinary perspectives on solving global problems.


\textbf{University Exchange Program}, Durham University, School of Applied Social Sciences, Durham, UK, 2017 - 2018 \\
\textit{Modules} \\
\begin{itemize}[leftmargin=*]
    \item Sociology of Work and Profession – Key concepts and theoretical frameworks in the sociology of work \\
    \item Social Policy: Principles and Current Issues – Understanding social policies and ideas that constitute these policies \\
    \item Research Methods in Action – Lectures on research design, qualitative method and quantitative method, and actual research projects. \\
    \item Economics of Social Policy – Applying principles of economics to areas of social policy and evaluate social policies from an economic standpoint \\
    \item Microeconomics – Intermediate microeconomics \\
\end{itemize}

An exchange program has developed my understanding of sociology especially from the perspective of economics. In addition, sociology modules have given me solid understandings of inequality, social class and stratification through social policy and sociology of profession, and quantitative and qualitative research methods in the context of actual situations.

    
\textbf{Degree}, International Honors Program, Intensive Studies, International Management Stanford University, Stanford, USA, 24 June 2017 - 20 August 2017 \\
\textit{Modules} \\
\begin{itemize}[leftmargin=*]
    \item Introduction to Decision Making – Analysis of the process of decision making based on probability theory and risk analysis. \\
    \item Comparative Corruption – Student-led presentation and discussion on concepts of political corruption and on case studies in developed and developing countries. \\
    \item Accounting for Managers and Entrepreneurs – Financial and managerial accounting, design of accounting system, techniques of analysis and cost control. \\
\end{itemize}


\textbf{Degree}, Summer School, Social Sciences Track, Sciences Po, Paris, France, 1 July 2016 - 29 July 2016 \\
\textit{Modules} \\
\begin{itemize}[leftmargin=*]
    \item  Comparative Perspectives on Migration Issues – Discussion of current migration issues based on political theories and case studies. \\
\end{itemize}

% Research Experience
\section*{Research Experience}
\begin{itemize}[leftmargin=*]
    \item \textbf{Research Experiences 1}, \\
    \item The University of Tokyo, Department of Sociology, Tokyo, Japan \\
    \item 2019 - 2020 \\
    \item \textit{Supervisor:} Professor Name \\
    \item 1. Bachelor’s Dissertation Research under Prof Seiyama and Prof Shirahase (April 2019 - February 2020)
    60,000 words in Japanese, approx. 30,000 words in English
    Analysed mechanism of educational assortative mating in Japan from both quantitative and qualitative perspectives.
    Developed discussion of assortative mating on dating apps in Japan.
    \item \textbf{Research Experiences 2}, \\
    \item The University of Tokyo, Department of Sociology, Tokyo, Japan \\
    \item 2019 \\
    \item 2. Field Research of Student Retention on Secondary Education in Phnom Penh Cambodia, a part of the Global Education for Innovation and Leadership (GEfIL) Program
\end{itemize}

% Presentations
\section*{Presentations}
\begin{itemize}[leftmargin=*]
    \item \textbf{Presentation},  Presentation at National Taiwan University and University of Tokyo joint seminar, Taipei, Taiwan \\
    September 2020 \\
    'Educational Assortative Marriage in Japan.  Analysing the Differences in Mating and Dating.'
\end{itemize}

% Scholarships and Awards
\section*{Scholarships and Awards}
\begin{itemize}[leftmargin=*]
    \item \textbf{Degree}, Krone Award \\
    The University of Tokyo, Department of Sociology, Tokyo, Japan, March 2020 \\
    The award for the most distinguished dissertation in a single year.
    \item \textbf{Degree}, GLP-GEfIL Scholarship for Abroad Experience \\
    \item The University of Tokyo, Department of Sociology, Tokyo, Japan, xx 20xx \\
    \item \textbf{Degree}, Alumni Scholarship for Abroad Experience \\
    \item The University of Tokyo, Department of Sociology, Tokyo, Japan, July 2016 \\
    \item Awarded for Science Po summer school in 2016. \\
\end{itemize}

% Professional Experiences
\section*{Professional Experiences}
\begin{itemize}[leftmargin=*]
    \item \textbf{Degree}, Subject \\
    University Name, Year
\end{itemize}

% Technical Skills
\section*{Technical Skills}
\begin{itemize}[leftmargin=*]
    \item \textbf{Degree}, Subject \\
    University Name, Year
\end{itemize}

\end{document}

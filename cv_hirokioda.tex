\documentclass[11pt, a4paper]{article}
\usepackage[utf8]{inputenc}
\usepackage{geometry}
\usepackage{enumitem}

% Set page margins
\geometry{left=2.5cm, top=2.5cm, right=2.5cm, bottom=2.5cm}
% Define a new command for the section with two columns
\newcommand{\cvsection}[2]{
    \noindent\textbf{#1} & \textbf{#2} \\
    \multicolumn{2}{c}{} \\ % Adding a little space after the section title
}

\begin{document}

% Header
\begin{center}
{\huge \textbf{Hiroki Oda}} \\
{\large \textit{Department of Methodology, London School of Economics and Political Science}} \\
\texttt{h.oda@lse.ac.uk | www.yourwebsite.com} \\
\end{center}


\section*{Education}
\noindent\begin{tabular}{@{}p{0.75\textwidth}p{0.25\textwidth}}
\textbf{London School of Economics and Political Science} & \hfill Sep 2023. -- present \\
MSc Applied Social Data Science \\
\textit{Modules:} Computer Programming, Social Science Research Design, Managing and Visualising Data, Applied Machine Learning for Social Science, Causal Inference for Observational and Experimental Studies, Social Network Analysis \\
\end{tabular} \\

\noindent\begin{tabular}{@{}p{0.75\textwidth}p{0.25\textwidth}}
\textbf{The University of Tokyo} & \hfill 2015 -- 2020 \\
BA Sociology \\
\textit{Modules:} Principle of Sociology, Research Methods in Sociology, Quantitative analysis of inequality and social stratification, Sociological Theory, Sociology of science and technology, Dissertation \\
\end{tabular} \\

\noindent \textbf{BA Thesis:} Empirical Analysis of Educational Assortative Mating in Japan \\
\textit{Supervisors:} Prof. Sawako Shirahase, Prof. Kazuo Seiyama, Prof. Takeshi Deguchi \\
\textit{Best dissertation of the year} \\

\noindent The degree in Sociology has given me skills of quantitative research using SPSS and R, critical thinking towards existing theory and common sense, logical thinking and discussion in abstract and concrete ideas. A dissertation for BA is awarded as one of the best dissertations of the year. \\

\noindent \textbf{Programme outside of the degree:} Global Education for Innovation and Leadership (GEfIL) Programme \\
\textit{Modules:} \\
Independent Research Project – Research on Rohingya refugees in Japan \\
Global Leader Lecture Series – Lectures on problem-solving strategies by globally engaged speakers \\

\noindent The honours program at The University of Tokyo has given me the skill of doing acdemic research in English, the ability in doing projects with people from different backgrounds and transdisciplinary perspectives on solving global problems. \\

\noindent \textbf{University Exchange Program} \\
University of Durham, School of Applied Social Sciences, Durham, UK, 2017 - 2018 \\
\textit{Modules:} Sociology of Work and Profession, Social Policy, Research Methods in Action, Economics of Social Policy, Microeconomics \\

\noindent An exchange program has developed my understanding of sociology especially from the perspective of economics. In addition, sociology modules have given me solid understandings of inequality, social class and stratification through social policy and sociology of profession, and quantitative and qualitative research methods in the context of actual situations. \\

\noindent \textbf{Summer International Honors Program} \\
Intensive Studies, International Management Stanford University, Stanford, USA, 24 June 2017 - 20 August 2017 \\
\textit{Modules:} Introduction to Decision Making, Comparative Corruption, Accounting for Managers and Entrepreneurs \\

\noindent \textbf{Summer School} Summer School, Social Sciences Track, Sciences Po, Paris, France, 1 July 2016 - 29 July 2016 \\
\textit{Comparative Perspectives on Migration Issues} \\

% Research Experience
\section*{Research Experience}
\begin{itemize}[leftmargin=*]
    \item \textbf{Research Experiences 1}, \\
    \item The University of Tokyo, Department of Sociology, Tokyo, Japan \\
    \item 2019 - 2020 \\
    \item \textit{Supervisor:} Professor Name \\
    \item 1. Bachelor’s Dissertation Research under Prof Seiyama and Prof Shirahase (April 2019 - February 2020)
    60,000 words in Japanese, approx. 30,000 words in English
    Analysed mechanism of educational assortative mating in Japan from both quantitative and qualitative perspectives.
    Developed discussion of assortative mating on dating apps in Japan.
    \item \textbf{Research Experiences 2}, \\
    \item The University of Tokyo, Department of Sociology, Tokyo, Japan \\
    \item 2019 \\
    \item 2. Field Research of Student Retention on Secondary Education in Phnom Penh Cambodia, a part of the Global Education for Innovation and Leadership (GEfIL) Program
\end{itemize}

% Presentations
\section*{Presentations}
2020 "Educational Assortative Marriage in Japan. Analysing the Differences in Mating and Dating.", National Taiwan University and The University of Tokyo joint seminar

% Scholarships and Awards
\section*{Scholarships and Awards}
\subsection*{Scholarships for overseas exchange programmes}
\textbf{JASSO}
The University of Tokyo, Department of Sociology, Tokyo, Japan, July 2016 \\
Awarded for exchange programme at Durham University \\
\textbf{GLP-GEfIL Scholarship for Abroad Experience}
The University of Tokyo, Department of Sociology, Tokyo, Japan, xx 20xx \\
\textbf{Alumni Scholarship for Abroad Experience}
The University of Tokyo, Department of Sociology, Tokyo, Japan, July 2016 \\
Awarded for Science Po summer school in 2016 \\
\subsection*{Award for undergraduate dissertation}
\textbf{Krone Award}
The University of Tokyo, Department of Sociology, Tokyo, Japan, March 2020 \\
The award for the most distinguished dissertation in a single year.

% Professional Experiences
\section*{Professional Experiences}
\begin{itemize}[leftmargin=*]
    \item \textbf{Degree}, Subject \\
    University Name, Year
\end{itemize}

% Technical Skills
\section*{Technical Skills}
\begin{itemize}[leftmargin=*]
    \item \textbf{Degree}, Subject \\
    University Name, Year
\end{itemize}

\end{document}

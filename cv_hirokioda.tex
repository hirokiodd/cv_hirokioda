\documentclass{article}

\title{Hiroki Oda
(Department of Methodology, London School of Economics and Political Science)
(h.oda@lse.ac.uk)}

\usepackage[round, longnamesfirst, sort&compress]{natbib}
\usepackage[top=20truemm,bottom=20truemm,left=30truemm,right=30truemm]{geometry}

\begin{document}

\maketitle

\section{Education}
The University of Tokyo, Department of Sociology, Tokyo, Japan      2015 - 2020
1. BA Sociology	                                                                                                           GPA: 3.64
<Module>
Principle of Sociology - Introductory lectures on sociology 
Research Methods in Sociology - Focused on quantitative methods 
Seminar in Sociology - Quantitative analysis of Japanese social stratification
History of Sociological Theories - Introduction of sociological theories
Topics in Sociology - Sociology of science technology and research in action
Dissertation - Educational Assortative Mating in Japan (Described in Research Experience)
The degree in Sociology has given me skills of quantitative research using SPSS and R, critical thinking towards existing theory and common sense, logical thinking and discussion in abstract and concrete ideas. A dissertation for BA is awarded as one of the best dissertations of the year.

2. Honours Program
Global Education for Innovation and Leadership (GEfIL) Program               2016 - 2020
<Module>
Independent Research Project – Research on Rohingya refugees in Japan
Global Leader Lecture Series – Lectures on problem-solving strategies by globally engaged speakers 
The honours program at The University of Tokyo has given me the skill of doing acdemic research in English, the ability in doing projects with people from different backgrounds and transdisciplinary perspectives on solving global problems.

Durham University, School of Applied Social Sciences, Durham, UK
2017 - 2018
University Exchange Program
<Module>
Sociology of Work and Profession – Key concepts and theoretical frameworks in the sociology of work
Social Policy: Principles and Current Issues – Understanding social policies and ideas that constitute these policies
Research Methods in Action – Lectures on research design, qualitative method and quantitative method, and actual research projects.
Economics of Social Policy – Applying principles of economics to areas of social policy and evaluate social policies from an economic standpoint
Microeconomics – Intermediate microeconomics
An exchange program has developed my understanding of sociology especially from the perspective of economics. In addition, sociology modules have given me solid understandings of inequality, social class and stratification through social policy and sociology of profession, and quantitative and qualitative research methods in the context of actual situations.


Stanford University, Stanford, USA                        24 June 2017 - 20 August 2017
International Honors Program, Intensive Studies, International Management
<Module>
Introduction to Decision Making – Analysis of the process of decision making based on probability theory and risk analysis.
Comparative Corruption – Student-led presentation and discussion on concepts of political corruption and on case studies in developed and developing countries.
Accounting for Managers and Entrepreneurs – Financial and managerial accounting, design of accounting system, techniques of analysis and cost control. 


Sciences Po, Paris, France	                                     1 July 2016 - 29 July 2016 
Summer School, Social Sciences Track
<Module>
Comparative Perspectives on Migration Issues – Discussion of current migration issues based on political theories and case studies.


\section{Research Experience}
The University of Tokyo, Department of Sociology, Tokyo, Japan      2019 - 2020
1. Bachelor’s Dissertation Research under Prof Seiyama and Prof Shirahase (April 2019 - February 2020)
60,000 words in Japanese, approx. 30,000 words in English
Analysed mechanism of educational assortative mating in Japan from both quantitative and qualitative perspectives.
Developed discussion of assortative mating on dating apps in Japan.

2. Field Research of Student Retention on Secondary Education in Phnom Penh Cambodia, a part of the Global Education for Innovation and Leadership (GEfIL) Program (March 2019)
Research on the transition from lower secondary education to upper secondary education in Cambodia.
Analysed mechanism of student retention using qualitative methods (semi-structured interview and thematic analysis)

\section{Presentations}
Presentation at National Taiwan University・University of Tokyo joint seminar, Taipei, Taiwan                                                                                   September 2020
'Educational Assortative Marriage in Japan.  Analysing the Differences in Mating and Dating.'


\section{Scholarships and Awards}
University of Tokyo Krone Award                                                           March 2020
The award for the most distinguished dissertation in a single year.

University of Tokyo GLP-GEfIL Scholarship for Abroad Experience           xx 20xx

University of Tokyo Alumni Scholarship for Abroad Experience           July 2016
Awarded for Science Po summer school in 2016.


\section{Professional Experiences}
CADDi Inc., Tokyo, Japan                                                      March 2022 –  Present
Data Analyst
Fulltime
Data analysis of product cost on manufacturing matching platform using Google Spreadsheet, Google BigQuery and Python libraries
Developing cost calculation algorithm of machining products using Python
Working in both Japan and overseas (e.g. Vietnam and Thailand)

Accenture Japan Ltd, Tokyo, Japan                            April 2020 –  February 2022
Business Strategy Consultant
Fulltime
Planned mid-term strategy for a Japanese manufacturing company and a Japanese electronics distributor.
Developed business models, operating models and business cases for clients' new businesses through data analysis of a large data set using Microsoft Excel, Microsoft Power BI and Microsoft Access.
Aside from strategy projects, I participated in  ‘Cloud First’ initiatives which aim for scaling cloud business, and got certifications of cloud engineering from various cloud vendors.

\section{Technical Skills}
Programming Languages Python, R, SQL
Data Analysis Tools Google BigQuery, Google Spreadsheet, Microsoft Excel
Systems Git(Hub), Google Cloud Platform (Certified Google Cloud Associate Cloud Engineer)


\end{document}
